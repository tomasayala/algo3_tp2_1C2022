\documentclass[titlepage,a4paper]{article}

\usepackage{a4wide}
\usepackage[colorlinks=true,linkcolor=black,urlcolor=blue,bookmarksopen=true]{hyperref}
\usepackage{bookmark}
\usepackage{fancyhdr}
\usepackage[spanish]{babel}
\usepackage[utf8]{inputenc}
\usepackage[T1]{fontenc}
\usepackage{graphicx}
\usepackage{float}

\pagestyle{fancy} % Encabezado y pie de página
\fancyhf{}
\fancyhead[L]{TP2 - G15}
\fancyhead[R]{Algoritmos y Programación III - FIUBA}
\renewcommand{\headrulewidth}{0.4pt}
\fancyfoot[C]{\thepage}
\renewcommand{\footrulewidth}{0.4pt}

\begin{document}
\begin{titlepage} % Carátula
	\hfill\includegraphics[width=6cm]{logofiuba.jpg}
    \centering
    \vfill
    \Huge \textbf{Trabajo Práctico 2 — Java}
    \vskip2cm
    \Large [7507/9502] Algoritmos y Programación III\\
    Grupo G15 \\
    Primer cuatrimestre 2022 \\
    \vfill
    Integrantes \\
    .\\
    \begin{tabular}{ | l | l | l | } % Datos del alumno
      \hline
      Alumno & Padrón & Email \\ \hline
      Alvaro Martin & 105040 & alvaro.martin1307@gmail.com \\ \hline
      Franco Macke & 105974 & fmacke@fi.uba.ar \\ \hline
      Tomás Ayala & 105336 & tayala@fi.uba.ar \\ \hline
      Guillermina Hoffmann & 104406 & ghoffmann@fi.uba.ar \\ \hline
      Álvaro Martín & 105040 & alvaro.martin1307@gmail.com \\ \hline
  	\end{tabular}
    \vfill
    \vfill
\end{titlepage}

\tableofcontents % Índice general
\newpage

\section{Introducción}\label{sec:intro}

\section{Supuestos}\label{sec:supuestos}
  \begin{itemize}
    \item Una calle puede tener más de un modificador
    \item Se le debe indicar al jugador donde inicia
    \item Asumimos que solo hay un jugador
    \item Las calles conectan 2 celdas
    \item El tablero es de dimensión variable
  \end{itemize}
\section{Diagramas de clase}\label{sec:diagramasdeclase}

Este es el esquema general de la aplicación. Más adelante, hay un esquema de secuencia donde se puede ver como interactuan entre si los objetos.


\begin{figure}[H]
  \centering
  \includegraphics[width=1\textwidth]{diagramas/modelo-inicial.png}
  \caption{\label{fig:seq01} Diagrama de clase general}
\end{figure}

\begin{figure}[H]
  \centering
  \includegraphics[width=1\textwidth]{diagramas/ClaseDireccion.png}
  \caption{\label{fig:seq02} Diagrama de clase direccion}
\end{figure}

\begin{figure}[H]
  \centering
  \includegraphics[width=1\textwidth]{diagramas/ClaseModificador.png}
  \caption{\label{fig:seq03} Diagrama de clase modificador}
\end{figure}

\begin{figure}[H]
  \centering
  \includegraphics[width=1\textwidth]{diagramas/ClaseVehiculo.png}
  \caption{\label{fig:seq04} Diagrama de clase vehículo}
\end{figure}

\begin{figure}[H]
\centering
% \includegraphics[width=0.8\textwidth]{diagrama_clase01.png}
\caption{\label{fig:class01}Explicacion diagrama}
\end{figure}

\section{Detalles de implementación}\label{sec:implementacion}

\subsubsection[Tablero]{Tablero}

La clase Tablero se encarga del manejo del juego. Delega su comportamiento a las clases correspondientes para el desarrollo.

\subsubsection[Modificador]{Modificador}

Esta interfaz está formada por el método “cruzarCon(Jugador jugador)”. Este método es utilizado por una instancia de la clase Calle para comunicarse con una clase que implementa Modificador. 
Las clases que implementan esta interfaz son: Piquete, Favorable, Desfavorable, Nulo, Pozo, ControlPolicial y CambioDeVehiculo. En cada una de estas clases se implementa el método de la siguiente forma:

\begin{itemize}
  \item Nulo: implementa el método de forma que solo se suma un movimiento al jugador. Utiliza la instancia de jugador que recibe por argumento y llama al método “sumarMovimientos(int numero)”. 
  \item Pozo: el método además de sumar un movimiento al jugador, por medio de la instancia de jugador obtiene el vehículo de este con el método “get.vehiculo()” y luego con esta instancia de vehículo llama al método pozo(Jugador jugador). 
  \item Piquete: el método además de sumar un movimiento al jugador, por medio de la instancia de jugador obtiene el vehículo de este con el método “get.vehiculo()” y luego con esta instancia de vehículo llama al método “piquete(Jugador jugador)”. 
  \item ControlPolicial: el método además de sumar un movimiento al jugador, por medio de la instancia de jugador obtiene el vehículo de este con el método “get.vehiculo()” y luego con esta instancia de vehículo llama al método “controlPolicial(Jugador jugador)”. 
  \item CambioDeVehiculo: el método además de sumar un movimiento al jugador, por medio de la instancia de jugador llama al método “reemplazarVehiculo()”.
  \item Desfavorable: el método además de sumar un movimiento al jugador, por medio de la instancia de jugador llama al método “sorpresaDesfavorable()”.
  \item Favorable: el método además de sumar un movimiento al jugador, por medio de la instancia de jugador llama al método “sorpresaFavorable()”.
  \item Todas las clases utilizan el método “sumarMovimientos(int numero)” de la clase Jugador para sumar un movimiento. 
\end{itemize}

\subsubsection[Jugador]{Jugador}

Al aplicarse una modificación al jugador, sabe como responder al mensaje que se le envía.


\subsubsection[Calle]{Calle}

La clase Calle guarda las referencias de 2 celdas, generando una conexión entre ellas. Cuando el jugador cruza por la calle, Calle le envía el respectivo mensaje.


% \section{Excepciones}\label{sec:excepciones}
% % Explicación de cada una de las excepciones creadas y con qué fin fueron creadas.

% \begin{description}
% \item[Exception] Explicacion excepcion 1
% \item[Excepcion] Explicacion excepcion 2
% \end{description}

\section{Diagramas de secuencia}\label{sec:diagramasdesecuencia}
% Mostrar las secuencias interesantes que hayan implementado. Pueden agregar texto para explicar si algo no queda claro.

Introducción


\subsubsection[Una moto atraviesa la ciudad y se encuentra con un Pozo. Es penalizada en tres movimientos]{Una moto atraviesa la ciudad y se encuentra con un Pozo. Es penalizada en tres movimientos}

\begin{figure}[H]
  \centering
  \includegraphics[width=1\textwidth]{diagramas/SecuenciaUnaMotoCruzaUnPozoYEsPenalizado.png}
  \caption{\label{fig:class01}Describe el caso donde una moto se cruza un pozo y recibe una penalización de 3 movimientos}
\end{figure}

\subsubsection[Un auto atraviesa la ciudad y se encuentra con un Pozo. Es penalizada en tres movimientos]{Un auto atraviesa la ciudad y se encuentra con un Pozo. Es penalizada en tres movimientos}

\begin{figure}[H]
  \centering
  \includegraphics[width=1\textwidth]{diagramas/SecuenciaAutoCruzaUnPozoYEsPenalizado.png}
  \caption{\label{fig:class01} Describe el caso donde un auto atraviesa la ciudad y se encuentra un pozo, recibiendo una penalización de 3 movimientos}
\end{figure}


\subsubsection[Una 4x4 atraviesa la ciudad y se encuentra con un Pozo. No es penalizada]{Una 4x4 atraviesa la ciudad y se encuentra con un Pozo. No es penalizada}

\begin{figure}[H]
  \centering
  \includegraphics[width=1\textwidth]{diagramas/SecuenciaUnaCamionetaCruzaUnPozoYNoEsPenalizado.png}
  \caption{\label{fig:class01} Acá se ve el caso donde una camioneta se cruza con un pozo. No se ve afectado la cantidad de movimientos}
\end{figure}



\subsubsection[Un auto cambia de vehiculo]{Un auto cambia de vehiculo}

\begin{figure}[H]
  \centering
  \includegraphics[width=1\textwidth]{diagramas/SecuenciaAutoCambiaVehiculo.png}
  \caption{\label{fig:class01}Explicacion diagrama}
\end{figure}



\subsubsection[Una moto cambia de vehiculo]{Una moto cambia de vehiculo}

\begin{figure}[H]
  \centering
  \includegraphics[width=1\textwidth]{diagramas/SecuenciaMotoCambiaVehiculo.png}
  \caption{\label{fig:class01}Explicacion diagrama}
\end{figure}


\end{document}